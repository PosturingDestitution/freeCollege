\documentclass{article}
\usepackage{enumitem}

\title{CCNA Study Guide}
\author{Alexander}
\date{\today}

\begin{document}
\maketitle

\begin{enumerate}
  \item Network Fundamentals (20\%)
  \item Network Access (20\%)
  \item IP Connectivity (25\%)
  \item IP Services (10\%)
  \item Security Fundamentals (15\%)
  \item Automation and Programmability (10\%)\\
\end{enumerate}



\begin{flushleft}\textbf{Network Fundamentals}\end{flushleft}
\begin{itemize}
  \item Explain the role and function of network components
  
  	\item[] Routers forward packets between computer networks (LANs and WANs), and use IP addresses to determine where the packet should go.

	\item[] L2 switches forward data frames within the same network segment using MAC addresses. L3 switches can perform switching functions in addition to routing functions.

	\item[] Firewalls sit in the path that packets take through the network. They permit/deny traffic much like an ACL would do on a router. Firewalls are capable of watching application-layer flows with AVC, performing webpage verification on URIs, and retaining state. IPSs can compare packet flows to exploit signatures, log events, and can discard/redirct packets.

  \item Describe characteristics of network topology architectures
  \item Compare physical interface and cabling types
  \item Identify interface and cable issues (collisions, error, mismatch duplex, and/or speed)
  \item Compare TCP to UDP
  \item Configure and verify IPv4 addressing and subnetting
  \item Describe private IPv4 addressing
  \item Describe IPv6 addressing and prefix
  \item Verify IP parameters for Client OS
  \item Describe wireless principles
  \item Explain virtualization fundamentals (server virtualization, containers, and VRFs)
  \item Describe switching concepts\\
\end{itemize}



\begin{flushleft}\textbf{Network Access}\end{flushleft}
\begin{itemize}
  \item Configure and verify VLANs (normal range) spanning multiple switches
  \item Configure verify interswitch connectivity
  \item Configure and verify L2 discovery protocols (CDP/LLDP)
  \item Configure and verify (L2/L3) EtherChannel (LACP)
  \item Interpret basic operations of Rapid PVST+ Spanning Tree Protocol
  \item Describe Cisco Wireless Architectures and AP modes
  \item Describe physical infrastructure connections of WLAN components (AP, WLC, access/trunk ports, and LAG)
  \item Describe network device management access (Telnet, SSH, HTTP, HTTPS, console, TACACS+/RADIUS, and cloud managed)
  \item Interpret the wireless LAN GUI configuration for client connectivity, such as WLAN creation, security settings, QoS profiles, and advanced settings\\
\end{itemize}



\begin{flushleft}\textbf{IP Connectivity}\end{flushleft}
\begin{itemize}
  \item Interpret the compnents of routing table
  \item Determine how a router makes a forwarding decision by default
  \item Configure and verify IPv4 and IPv6 static routing 
  \item Configure and verify single area OSPFv2
  \item Describe the purpose, functions, and concepts of first hop redundancy protocols\\
\end{itemize}



\begin{flushleft}\textbf{IP Services}\end{flushleft}
\begin{itemize}
  \item Configure and verify inside source NAT using static and pools
  \item Configure and verify NTP operating in a client and server mode
  \item Explain the role of DHCP and DNS within the network
  \item Explain the function of SNMP in network operations
  \item Describe the use of syslog features including facilities and levels
  \item Configure and verify DHCP client and relay
  \item Explain the forwarding per-hop behavior (PHB) for QoS, such as classification, marking, queuing, congestion, policing, and shaping
  \item Configure network devices for remote access using SSH
  \item Describe the capabilities and functions of TFTP/FTP in the network\\
\end{itemize}



\begin{flushleft}\textbf{Security Fundamentals}\end{flushleft}
\begin{itemize}
  \item Define key security concepts (threats, vulnerabilities, exploits, and mitigation techniques)
  \item Describe security program elements (user awareness, training, and physical access control)
  \item Configure and verify device access control using local passwords
  \item Describe security password policies elements, such as management, complexity, and password alternatives (multifactor authentication, certificates, and biometrics)
  \item Describe IPsec remote access and site-to-site VPNs
  \item Configure and verify access control lists
  \item Configure and verify Layer 2 security features (DHCP snooping, dynamic ARP inspection, and port security)
  \item Compare authentication, authorization, and accounting concepts
  \item Describe wireless security protocols (WPA, WPA2, and WPA3)
  \item Configure and verify WLAN within the GUI using WPA2 PSK\\
\end{itemize}



\begin{flushleft}\textbf{Automation and Programmability}\end{flushleft}
\begin{itemize}
  \item Explain how automation impacts network management
  \item Compare traditional networks with controller-based networking
  \item Describe controller-based, software defined architecture (overlay, underlay, and fabric) 
  \item Explain AI (generative and predictive) and machine learning in network operations
  \item Describe characteristics of REST-based APIs (authentication types, CRUD, HTTP verbs, and data encoding)
  \item Recognize the capabilities of configuration management mechanisms, such as Ansible and Terraform
  \item Recognize components of JSON-encoded data\\
\end{itemize}

\end{document}
